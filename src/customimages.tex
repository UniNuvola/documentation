\section{Anatomy of the UniNuvola Images system}\label{image_anatomy}
\uninuvola's images are a collection of container images recipes that are
available in the UniNuvola's GitHub \href{https://github.com/UniNuvola/images}
{images repository}. 
These images have been created by the UniNuvola team and are available for all users.
The base of all images is the jupyter notebook image, a system widely used in the
scientific community, extended with many packages to cover a wide range of
scientific applications commonly used in our University.

The images can be seen as a tree structure, where the root is the base image
``base'' and the branches adds new features to the base image and eventually 
are the origin of different leaves. 
For example, the image ``base/conda`` brings the conda package manager to the base
image, while ``base/conda/pytorch`` adds the pytorch package to the conda image.

With this in mind, building a new image is as simple as copying one of the
available recipes, modifying it to your needs and building a new image.
To ease this process, we have created a public template for the images creation.

\section{Re-creation of the base image}\label{image_recreation}
Let's see how to create a replica of the base image with a different name and belonging to the user.

\begin{bclogo}[logo=\bcinfo, couleurBarre=orange, noborder=true, couleur=white]{Warning}
  Pre-requisite: you need to have a github account to replicate the base
  image. You can use your personal account or create a new one.
\end{bclogo}

Login to your github account and go to the UniNuvola's \href{https://github.com/UniNuvola/custom-image-template}{custom image template}
repository. Click on the ``Use this template'' button to create a new
repository based on the template. This will create a new repository in your
github account with the same structure as the template. You can name it
any name you like. \\
This simple action trigger a github action that will automatically build
the image and push it to the GitHub registry. The action will take a few
minutes to complete. You can check the progress of the action by going to the ``Actions'' tab of
your new repository. \\
Once the action is completed, you will see a new image in your GitHub
repository. You can check the image by going to the ``Packages'' tab of your
repository.
\\
By clicking on the image name, you will see the details of the image, including
the full name of the image, that should be in the format: \\
\\
``ghcr.io/\textit{your\_github\_username}/\textit{your\_repository\_name}:main''
\\
\\
You can use this name to run the image in your UniNuvola environment, by
putting it in the ``Select your desired image'' field of the
\jupyterhubn{} interface. \\
\section{Customizing the image}\label{image_customization}
The main part of the image creation is done by modifying the Dockerfile. The
Dockerfile is a text file that contains all the instructions to build the
Docker image. The Dockerfile is located in the root directory of your
repository and contains a placeholder for the custom code:
\begin{lstlisting}[language=python]
  ## -- ADD YOUR CODE HERE !! -- ##

  ## --------------------------- ##
\end{lstlisting}
This is the place where you can add your custom code. You can add any
Dockerfile command you like. Take a look at the \href{https://github.com/UniNuvola/images}
{UniNuvola images repository} to see some examples of Dockerfiles we have
created. \\
You can either modify the file directly in the github interface or clone the
repository to your local machine and modify it there. If you choose the
latter option, you will need to push the changes to your github repository
after you are done. \\
In any case, every time you push a change to the repository, the github action will
automatically build the image and push it to the GitHub registry, as described in
the previous section. \\
Cloning the repository to your local machine is a good option since you 
can use your favorite text editor to modify the Dockerfile. You can also
build the image locally and check if it builds correctly before pushing it to
github. To do this, you will need to have a container engine installed on
your local machine, like Docker. \\
Describing how to install and use locally a container runtime is out of the scope of this document.

\subsection{Some examples of Dockerfile commands}

The most straightforward way to learn how to modify the Dockerfile is to
look at the existing images in the \href{https://github.com/UniNuvola/images}
{UniNuvola images repository}.

You can add any command you like to the Dockerfile. The most common commands
are:
\begin{itemize}
  \item \textit{Installing packages}: You can install new packages using the
  appropriate package managers, using the \textit{RUN} keyword in the Dockerfile. \\
  For example, to install a package using \textit{apt}, you can use the following

\begin{lstlisting}[language=python]
  RUN apt-get update && apt-get install -y package\_name
\end{lstlisting}

  \item \textit{Copying files}: You can copy files from your local machine to the
  image using the \textit{COPY} command. \\
  For example, to copy a file called ``my\_file.txt'' from your local machine
  to the image, you can use the following command:
\begin{lstlisting}[language=python]
  COPY my\_file.txt /path/in/image/
\end{lstlisting}

  \item \textit{Setting environment variables}: You can set environment
  variables using the \textit{ENV} command. \\
  For example, to set an environment variable called ``MY\_VAR'' to the value
  ``my\_value'', you can use the following command:
\begin{lstlisting}[language=python]
  ENV MY\_VAR=my\_value
\end{lstlisting}

\end{itemize}
