UniNuvola is a cloud computing platform designed to provide users with a flexible and powerful environment for running scientific applications.
It is built on top of the JupyterHub platform, which allows multiple users to access and run Jupyter notebooks in a shared environment.
The platform is designed to be user-friendly and accessible, with a focus on providing a seamless experience for users of all levels of expertise.

The origin of UniNuvola came from the collaborative actions of the University of Perugia within the Work Package 4.4 , High Performance Computing.
The goal of this work package is to provide a proof of concept (and a prototype) on how the University could provide a cloud computing platform to its researchers and students.

For this reason, UniNuvola is a prototype and it is not intended to be used for production purposes. The platform is designed to be used for testing and experimentation, and users should be aware that the platform may not be stable or reliable.

This document is intended to provide a comprehensive guide to using UniNuvola, including instructions for logging in, selecting resources, and running applications.
It also includes information on how to create custom images and manage resources, as well as troubleshooting tips and best practices for using the platform.

The document is divided into several parts, each of which covers a different aspect of using UniNuvola.

\begin{itemize}
	\item \textbf{Quickstart Guide:} The first part is a quickstart guide for the impatient users, who want to start using the platform as soon as possible.
	\item \textbf{System Access:} The second part covers the system access, including how to log in and select resources.
	\item \textbf{Working with UniNuvola:} The third part covers how to work with UniNuvola, including how to run applications and manage resources.
	\item \textbf{Advanced Usage:} The fourth part covers advanced usage, including how to create custom images and manage resources.
	\item \textbf{Troubleshooting:} The fifth part covers troubleshooting tips and best practices for using the platform.
	\item \textbf{Appendix:} The last part includes an appendix with additional information and resources.
\end{itemize}
 
 





