\section{Anatomy of the UniNuvola Images system}\label{image_anatomy}
\uninuvola's images are a collection of container images recipes that are
available in the UniNuvola's GitHub \href{https://github.com/UniNuvola/images}
{images repository}. 
These images have been created by the UniNuvola team and are available for all users.
The base of all images is the jupyter notebook image, a system widely used in the
scientific community, extended with many packages to cover a wide range of
scientific applications commonly used in our University.

The images can be seen as a tree structure, where the root is the base image
``base'' and the branches adds new features to the base image and eventually 
are the origin of different leaves. 
For example, the image ``base/conda`` brings the conda package manager to the base
image, while ``base/conda/pytorch`` adds the pytorch package to the conda image.

With this in mind, building a new image is as simple as copying one of the
available recipes, modifying it to your needs and building a new image.
To ease this process, we have created a public template for the images creation.

\section{Re-creation of the base image}\label{image_recreation}
Let's see how to create a replica of the base image with a different name and belonging to the user.

TODO from here

\section{Creation of an image}\label{image_creation} All \uninuvola's Docker
images are built on top of the ``Default'' image. You can copy one of the
available  Dockerfile in your project directory and start modifying it. You can
install new packages using the appropriate package managers, using the
\textit{RUN} keyword in the Dockerfile

\begin{lstlisting}[language=python]
  RUN apt-get update && apt-get install -y package\_name
\end{lstlisting}

Copy your application code into the Docker image, using the \textit{COPY} (for
local files) or \textit{ADD} (for local and remote files) command in the
Dockerfile. Do not use the /home directory as the target directory as it will be
overwritten when loading the external storage.

\begin{lstlisting}[language=python]
  COPY myfile.txt  /app ADD https://example.com/path/to/remote/file.txt /app/
\end{lstlisting}
%5. Set Environment Variables (if needed): If your application relies on
%    environment variables, set them using the ENV keyword in your Dockerfile.
%    For example: ENV ENV_VARIABLE=value

In order to build your Docker Imaga, open a terminal and navigate to the
directory containing the Dockerfile. Run the following command to build your
Docker image:

\begin{lstlisting}[language=python]
    docker build -f image\_name -t image\_tag .
\end{lstlisting}

The \textit{-f} tag allows you to name your dockerfile with a custom name, while
the \textit{-t}  flag allows the user to define a customised tag for your image.
Replace image\_name with a suitable name for your Docker image. \\

Once the image is built successfully, you can run a Docker container using the
following command:

\begin{lstlisting}[language=python]
    docker run -it image\_tagname 
\end{lstlisting}

The \textit{-it} tag allows you to test your image locally and interactively. \\

You can include a .dockerignore file to specify files and directories that
should not be copied into the Docker image.