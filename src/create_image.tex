\section{Creation of an image}\label{image_creation} All \uninuvola's Docker
images are built on top of the ``Default'' image. You can copy one of the
available  Dockerfile in your project directory and start modifying it. You can
install new packages using the appropriate package managers, using the
\textit{RUN} keyword in the Dockerfile

\begin{lstlisting}[language=python]
  RUN apt-get update && apt-get install -y package\_name
\end{lstlisting}

Copy your application code into the Docker image, using the \textit{COPY} (for
local files) or \textit{ADD} (for local and remote files) command in the
Dockerfile. Do not use the /home directory as the target directory as it will be
overwritten when loading the external storage.

\begin{lstlisting}[language=python]
  COPY myfile.txt  /app ADD https://example.com/path/to/remote/file.txt /app/
\end{lstlisting}
%5. Set Environment Variables (if needed): If your application relies on
%    environment variables, set them using the ENV keyword in your Dockerfile.
%    For example: ENV ENV_VARIABLE=value

In order to build your Docker Imaga, open a terminal and navigate to the
directory containing the Dockerfile. Run the following command to build your
Docker image:

\begin{lstlisting}[language=python]
    docker build -f image\_name -t image\_tag .
\end{lstlisting}

The \textit{-f} tag allows you to name your dockerfile with a custom name, while
the \textit{-t}  flag allows the user to define a customised tag for your image.
Replace image\_name with a suitable name for your Docker image. \\

Once the image is built successfully, you can run a Docker container using the
following command:

\begin{lstlisting}[language=python]
    docker run -it image\_tagname 
\end{lstlisting}

The \textit{-it} tag allows you to test your image locally and interactively. \\

You can include a .dockerignore file to specify files and directories that
should not be copied into the Docker image.